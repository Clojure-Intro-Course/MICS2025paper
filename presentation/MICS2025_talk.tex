\documentclass{beamer}
\usepackage{beamerthemeshadow}
\usepackage{color}
\usepackage[all]{xy}


%%%% choose your presentation style:
\mode<presentation>
{
  \usetheme{Copenhagen} %%%
 \usecolortheme{beaver}

%%% set style for ovelays: lists (and other text) appearing one item at a time
%%% This will create a dimmed preview of next item:
\setbeamercovered{transparent}
%%% This will hide it entirely:
%\setbeamercovered{invisible}
}
%% if you don't want page numbers to show: 
\setbeamertemplate{footline}[page number]{}

%%%% 25 minutes time slot, including questions 

\begin{document}
\title{A beginner-friendly environment for exploring error messages in the Clojure programming language.}
\author{Tristan Kalvoda, Elena Machkasova, Jaydon Stanislowski, and John Walbran}
\institute[UMN Morris] % (optional, but mostly needed)
{
 % \inst{1}%
  University of Minnesota, Morris
}
\date[]  
{Midwest Instruction and Computing Symposium, April 2025}

\begin{frame}
  \titlepage
\end{frame}

\begin{frame}

  \frametitle{Outline}
\tableofcontents
\end{frame}

\section{Overview of Clojure and Its Error Messages}

\begin{frame}
\frametitle{Tristan's section}

\end{frame}

\section{Babel project}

\subsection{Setup and Goals}

\begin{frame}
\frametitle{Jaydon's section}

\end{frame}

\subsection{Exceptions Processing}

\begin{frame}
\frametitle{Jaydon's section}

\end{frame}

\section{Morse Viewers}

\begin{frame}
\frametitle{John's section}

John's section
\end{frame}

\section{Current State of the Project and Future Work}

\begin{frame}
\frametitle{John's section}

John's section
\end{frame}

\begin{frame}
  \frametitle{Discussion}
Questions?
\end{frame}

\end{document}
