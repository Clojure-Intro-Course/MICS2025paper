\documentclass{beamer}
\usepackage{beamerthemeshadow}
\usepackage{color}
\usepackage[all]{xy}

%% Elena's favorite green (thanks, Fernando!)
\definecolor{ForestGreen}{RGB}{34,139,34}
\definecolor{BlueViolet}{RGB}{138,43,226}
\definecolor{Coquelicot}{RGB}{255, 56, 0}
\definecolor{Teal}{RGB}{2,132,130}
%Uncomment this if you want to show work-in-progress comments
\newcommand{\comment}[1]{{\bf \tt  {#1}}}
% Uncomment this if you don't want to show comments
%\newcommand{\comment}[1]{}
% for color names see https://www.overleaf.com/learn/latex/Using_colors_in_LaTeX
\newcommand{\emcomment}[1]{\textcolor{ForestGreen}{\comment{Elena: {#1}}}}
\newcommand{\todo}[1]{\textcolor{blue}{\comment{To Do: {#1}}}}
\newcommand{\tkcomment}[1]{\textcolor{Teal}{\comment{Tristan: {#1}}}}
\newcommand{\jscomment}[1]{\textcolor{olive}{\comment{Jaydon: {#1}}}}
\newcommand{\jwcomment}[1]{\textcolor{violet}{\comment{John: {#1}}}}
%%%%%%%%%%%%%%%%%%%%%%%%%%%%%%%%%%%%%%%%%%


%%%% choose your presentation style:
\mode<presentation>
{
  \usetheme{Copenhagen} %%%
 \usecolortheme{beaver}

%%% set style for ovelays: lists (and other text) appearing one item at a time
%%% This will create a dimmed preview of next item:
\setbeamercovered{transparent}
%%% This will hide it entirely:
%\setbeamercovered{invisible}
}
%% if you don't want page numbers to show: 
\setbeamertemplate{footline}[page number]{}

%%%% 25 minutes time slot, including questions 

\begin{document}
\title{A beginner-friendly environment for exploring error messages in the Clojure programming language.}
\author{Tristan Kalvoda, Elena Machkasova, Jaydon Stanislowski, and John Walbran}
\institute[UMN Morris] % (optional, but mostly needed)
{
 % \inst{1}%
  University of Minnesota, Morris
}
\date[]  
{Midwest Instruction and Computing Symposium, April 2025}

\begin{frame}
  \titlepage
\end{frame}

\begin{frame}

  \frametitle{Outline}
\tableofcontents
\end{frame}

\section{Overview of Clojure and Its Error Messages}

\subsection{Clojure language and Syntax}
\begin{frame}
\frametitle{Clojure Language and Syntax}
What is Clojure? - Clojure language and Syntax
\begin{itemize}
  \item Clojure is a part of the Lisp language family
  \item Syntax
  \begin{itemize}
    \item prefix notation (operators before operands).
    \item expressions are surrounded by parentheses.
  \end{itemize}
\end{itemize}
Example: \texttt{(/ 9 3)} denotes 9 divided by 3
\end{frame}

\begin{frame}
  \frametitle{Clojure Language and Syntax}
  \begin{itemize}
    \item Clojure \emcomment{is implemented in Java and} runs on the Java Virtual Machine (JVM)
    \begin{itemize}
      \item executed code compiles to JVM bytecode \emcomment{I corrected the line below slightly. Not sure if you need this line.}
    \end{itemize}
    \item Clojure code \(\rightarrow\) Java code \(\rightarrow\) JVM bytecode \(\rightarrow\) \\ executed on JVM
  \end{itemize}
\end{frame}

\begin{frame}
  \frametitle{Clojure Language and Syntax}
  Clojure's REPL
  \begin{itemize}
    \item interactive environment for code evaluation
    \item Read \(\rightarrow\) Evaluate \(\rightarrow\) Print \(\rightarrow\) Loop
    \tkcomment{repl example image}
  \end{itemize}  
\end{frame}

\subsection{Clojure's Error Messages}
\begin{frame}
  \frametitle{Clojure's Error Messages}
  Clojure Exceptions
  \begin{itemize}
    \item an event or error that disrupts the normal flow of a program's execution
    \item Clojure syntax errors will also result in an exception \emcomment{mention that it is a Java exception}
  \end{itemize}
  Error Messages
  \begin{itemize}
    \item generate when a exception occurs
    \item provide error type and location
  \end{itemize}
\end{frame}

\begin{frame}
    \frametitle{Clojure's Error Messages}
    Anatomy of a Clojure Error Message \\
    \texttt{=> (/ 9 0)} \\
    \texttt{Execution error (ArithmeticException) at user/eval1 (REPL:1).} \\
    \texttt{Divide by zero}
    \begin{itemize}
      \item \texttt{ArithmeticException}: The type of error that occurred.
      \item \texttt{user/eval1 (REPL:1)}: The location where the error happened (in this case, REPL, line 1).
      \item \texttt{Divide by zero}: The description of the error's cause.
    \end{itemize}
\end{frame}

\begin{frame}
  \frametitle{Clojure's Error Messages}
  \textbf{Exception Example} \\
  \texttt{\#error \{ } \\
  \texttt{:cause "Divide by zero"} \\
  \texttt{:via} \\
  \texttt{[{:type java.lang.ArithmeticException} } \\
  \texttt{:message "Divide by zero"} \\
  \texttt{:at [clojure.lang.Numbers divide "Numbers.java" 190]\}]} \\
  \texttt{:trace} \\
  \texttt{[[clojure.lang.Numbers divide "Numbers.java" 190]} \\
  \texttt{... omitting 18 lines...} \\
  \texttt{[clojure.main main "main.java" 40]]}
  \tkcomment{image instead? or maybe not add this}
\emcomment{I don't think you need this slide}
\end{frame}

\section{Babel project}

\subsection{Setup and Goals}
\begin{frame}
    \frametitle{Setup and Goals}
    Overview of Babel
    \begin{itemize}
        \item Tool designed to replace native Clojure messages to aid in understanding
        \item Relies heavily on the Clojure spec library to catch errors on function calls
\emcomment{We didn't introduce spec yet - can introduce it here; show "spec" and "other errors" in boldface or some such.}
        \item Maintains a dictionary of other errors (e.g. division by zero) that can't be spec'd, in order to rewrite them as well 
\emcomment{Using RegEx to pull out different parts}
    \end{itemize}
    Usage
    \begin{itemize}
        \item Launching a REPL server in the Babel repository allows the tool to "hook" to it
        \item Initialization function \texttt{(setup-exc)} is called to begin intercepting error messages
\emcomment{I don't think we need to mention this}
        \item All messages displayed in the terminal are generated by Babel rather than Clojure
    \end{itemize}
\end{frame}

\begin{frame}
    \frametitle{Setup and Goals}
    Motivation
\emcomment{shouldn't this be before the previous slide?}
    \begin{itemize}
        \item Babel is a learning tool for beginners to Clojure and programming as a whole
        \item Clojure error messages contain awkward phrasing that may impede understanding
    \end{itemize}
    \textbf{Example} \\
    Consider the error produced by the form below. What does it mean? \\
    \texttt{=> (count 1)} \\
    \texttt{Execution error (UnsupportedOperationException) at user/eval1529 (REPL:1).} \\
    \texttt{count not supported on this type: Long} \\
\end{frame}
    
\begin{frame}
\frametitle{Setup and Goals}

\end{frame}

\subsection{Exceptions Processing}

\begin{frame}
\frametitle{Exceptions Processing}

\end{frame}

\section{Morse Viewers}

\begin{frame}{Sending Data to Morse}
  \begin{itemize}
    \item<1-> The Clojure REPL does not provide the proper hooks to effectively manipulate error message data.
    \item<2-> To get around this, we need to initialize Babel within a sub-REPL of the parent REPL session.
    \item<3-> Creating a sub-REPL allows us to introduce hooks that let us add preprocessing steps.
  \end{itemize}
\end{frame}

\begin{frame}{Sub-REPL hooks}
  Babel uses the following hooks as part of error processing:
  \begin{itemize}
    \item<1-> \texttt{:init} Defines initial behavior on creation. In Babel this starts a new Morse session connected to the current REPL.
    \item<2-> \texttt{:eval} Defines behavior when a command is run. In Babel this stores the command verbatim into an atom, and evaluates the command in both the REPL and Morse.  
  \end{itemize}
\end{frame}

\begin{frame}{Sub-REPL hooks (cont.)}
  \begin{itemize}
  \item<1-> \texttt{:caught} Defines behavior on an exception. In Babel this processes the error, and passes the following information to Morse for display in a custom viewer:
  \begin{itemize}
    \item<2-> The last command entered, read from an atom that is updated at evaluation.
    \item<3-> The location in the environment where the error occurred. In the REPL, this
    resolves to “Clojure Interactive Session”.
    \item<4-> A vector of pairs containing the error message produced by Babel, with labels associated with each segment denoting its type for formatting.
    \item<5-> The url to the documentation of the function called that caused the error.
  \end{itemize}
\end{itemize}
\end{frame}

\section{Current State of the Project and Future Work}
\begin{frame}{Current State of the Project}
  \begin{itemize}
    \item<1-> We have existing error messages without labels for many common errors of core functions.
    \item<2-> We can connect Morse to a REPL session, and have mirroring form evaluation.
    \item<3-> Most of the work this year was spent structuring things for integration with Morse viewers.
    \item<4-> The introduction of the error labeling and prototyping this was pivotal in enabling data formatting.
    \item<5-> We currently have a small number of error messages labeled for demonstration purposes.
  \end{itemize}
\end{frame}

\begin{frame}{Future Work}
  \begin{itemize}
    \item<1-> Expand data labeling to all Babel error messages, to better format data for Morse viewers.
    \item<2-> Add hover text for specific terms to add definitions and supplementary information to the presented error message.
    \item<3-> Refining the end user work flow between working code and erroring code.
    \item<4-> Develop Morse viewers for other information, such as the stack trace, and full java error messages.
  \end{itemize}
  \end{frame}

\begin{frame}{Future Work (cont.)}
  \begin{itemize}
    \item<1-> We plan to run a usability study about our developments after we have greater feature coverage.
    \item<2-> We are going to use the results to guide further design.
    \item<3-> We hope to explore IDE integration for possible work-flow refinements.
    %Overview of the VSCode extension development world.
    %https://code.visualstudio.com/api
    %Webview display, allows us to reuse existing viewer code.
    %https://code.visualstudio.com/api/extension-guides/webview
    %Language protocol guide: (LSP, syntax highlighting, etc.)
    %https://code.visualstudio.com/api/language-extensions/overview
  \end{itemize}
\end{frame}

\begin{frame}{Acknowledgements}
This work was supported in part by Morris Academic Partnership (MAP) and UMN Undergraduate Research Opportunity (UROP).  \\ 

\vspace*{0.2in}

We thank Joe Lane for introducing us to Morse tools and for numerous helpful discussions.
\end{frame}

\begin{frame}
  \frametitle{Discussion}
Questions?
\end{frame}

\end{document}
