\documentclass{beamer}
\usepackage{beamerthemeshadow}
\usepackage{color}
\usepackage[all]{xy}

%% Elena's favorite green (thanks, Fernando!)
\definecolor{ForestGreen}{RGB}{34,139,34}
\definecolor{BlueViolet}{RGB}{138,43,226}
\definecolor{Coquelicot}{RGB}{255, 56, 0}
\definecolor{Teal}{RGB}{2,132,130}
%Uncomment this if you want to show work-in-progress comments
\newcommand{\comment}[1]{{\bf \tt  {#1}}}
% Uncomment this if you don't want to show comments
%\newcommand{\comment}[1]{}
% for color names see https://www.overleaf.com/learn/latex/Using_colors_in_LaTeX
\newcommand{\emcomment}[1]{\textcolor{ForestGreen}{\comment{Elena: {#1}}}}
\newcommand{\todo}[1]{\textcolor{blue}{\comment{To Do: {#1}}}}
\newcommand{\tkcomment}[1]{\textcolor{Teal}{\comment{Tristan: {#1}}}}
\newcommand{\jscomment}[1]{\textcolor{olive}{\comment{Jaydon: {#1}}}}
\newcommand{\jwcomment}[1]{\textcolor{violet}{\comment{John: {#1}}}}
%%%%%%%%%%%%%%%%%%%%%%%%%%%%%%%%%%%%%%%%%%


%%%% choose your presentation style:
\mode<presentation>
{
  \usetheme{Copenhagen} %%%
 \usecolortheme{beaver}

%%% set style for ovelays: lists (and other text) appearing one item at a time
%%% This will create a dimmed preview of next item:
\setbeamercovered{transparent}
%%% This will hide it entirely:
%\setbeamercovered{invisible}
}
%% if you don't want page numbers to show: 
\setbeamertemplate{footline}[page number]{}

%%%% 25 minutes time slot, including questions 

\begin{document}
\title{A beginner-friendly environment for exploring error messages in the Clojure programming language.}
\author{Tristan Kalvoda, Elena Machkasova, Jaydon Stanislowski, and John Walbran}
\institute[UMN Morris] % (optional, but mostly needed)
{
 % \inst{1}%
  University of Minnesota, Morris
}
\date[]  
{Midwest Instruction and Computing Symposium, April 2025}

\begin{frame}
  \titlepage
\end{frame}

\begin{frame}

  \frametitle{Outline}
\tableofcontents
\end{frame}

\section{Overview of Clojure and Its Error Messages}

\subsection{Clojure language and Syntax}
\begin{frame}
\frametitle{Clojure Language and Syntax}
What is Clojure? - Clojure language and Syntax
\begin{itemize}
  \item Clojure is a part of the lisp language family
  \item Syntax
  \begin{itemize}
    \item prefix notation (operators before operands).
    \item expressions are surrounded by parentheses.
  \end{itemize}
\end{itemize}
Example: \texttt{(/ 9 3)} denotes 9 divided by 3
\end{frame}

\begin{frame}
  \frametitle{Clojure Language and Syntax}
  \begin{itemize}
    \item Clojure runs on the Java Virtual Machine (JVM)
    \begin{itemize}
      \item executed code compiles to JVM bytecode
    \end{itemize}
    \item clojure code \(\rightarrow\) clojure compiler \(\rightarrow\) JVM bytecode \(\rightarrow\) executed on JVM
  \end{itemize}
\end{frame}

\begin{frame}
  \frametitle{Clojure Language and Syntax}
  Clojure's REPL
  \begin{itemize}
    \item interactive environment for code evaluation
    \item Read \(\rightarrow\) Evaluate \(\rightarrow\) Print \(\rightarrow\) Loop
    \tkcomment{repl example image}
  \end{itemize}  
\end{frame}

\subsection{Clojure's Error Messages}
\begin{frame}
  \frametitle{Clojure's Error Messages}
  Clojure Exceptions
  \begin{itemize}
    \item an event or error that disrupts the normal flow of a program's execution
    \item Clojure syntax errors will also result in an exception
  \end{itemize}
  Error Messages
  \begin{itemize}
    \item generate when a exception occurs
    \item provide error type and location
  \end{itemize}
\end{frame}

\begin{frame}
    \frametitle{Clojure's Error Messages}
    Anatomy of a Clojure Error Message \\
    \texttt{=> (/ 9 0)} \\
    \texttt{Execution error (ArithmeticException) at user/eval1 (REPL:1).} \\
    \texttt{Divide by zero}
    \begin{itemize}
      \item \texttt{ArithmeticException}: The type of error that occurred.
      \item \texttt{user/eval1 (REPL:1)}: The location where the error happened (in this case, REPL, line 1).
      \item \texttt{Divide by zero}: The description of the error's cause.
    \end{itemize}
\end{frame}

\begin{frame}
  \frametitle{Clojure's Error Messages}
  \textbf{Exception Example} \\
  \texttt{\#error \{ } \\
  \texttt{:cause "Divide by zero"} \\
  \texttt{:via} \\
  \texttt{[{:type java.lang.ArithmeticException} } \\
  \texttt{:message "Divide by zero"} \\
  \texttt{:at [clojure.lang.Numbers divide "Numbers.java" 190]\}]} \\
  \texttt{:trace} \\
  \texttt{[[clojure.lang.Numbers divide "Numbers.java" 190]} \\
  \texttt{... omitting 18 lines...} \\
  \texttt{[clojure.main main "main.java" 40]]}
  \tkcomment{image instead? or maybe not add this}
\end{frame}

\section{Babel project}

\subsection{Setup and Goals}

\begin{frame}
\frametitle{Jaydon's section}

\end{frame}

\subsection{Exceptions Processing}

\begin{frame}
\frametitle{Jaydon's section}

\end{frame}

\section{Morse Viewers}

\begin{frame}{Sending Data to Morse}
  \begin{itemize}
    \item<1-> The Clojure REPL does not provide the proper hooks to effectively manipulate error message data.
    \item<2-> To get around this, we need to initialize Babel within a sub-REPL of the parent REPL session.
    \item<3-> Creating a sub-REPL allows us to introduce hooks that let us add preprocessing steps.
  \end{itemize}
\end{frame}

\begin{frame}{Sub-REPL hooks}

\end{frame}

\section{Current State of the Project and Future Work}
\begin{frame}{Current State of the Project}
\emcomment{Mention older things that we have accomplished}
  \begin{itemize}
    \item<1-> Most of the work this year was spent structuring things for integration with Morse viewers.
    \item<2-> The introduction of the error labeling \emcomment{with labels like ....} and prototyping this was pivotal in enabling data formatting.
    \item<3-> We currently have a small number of error messages labeled for demonstration purposes.
  \end{itemize}
\end{frame}

\begin{frame}{Future Work}
  The following are areas of active development:
  \begin{itemize}
    \item<1-> Expand data labeling to all Babel error messages.
    \item<2-> Add hover text for specific terms to add definitions and supplementary information to the presented error message.
    \item<3-> Refining the end user work flow between working code and erroring code.
  \end{itemize}
\emcomment{Mention developing Morse viewers for specific labels and other info, such as stack trace and full Java error messages.}
  \end{frame}

\begin{frame}{Future Work (cont.)}
\emcomment{simplify the sentences}
  \begin{itemize}
    \item<1-> Once we have greater feature coverage in Babel, we plan to run a usability study about the interactive tools we have developed.
    \item<2-> We are going to use the results of letting users explore our tools while learning Clojure in order to guide further design. 
    \item<3-> We would like to explore IDE integration to further expand possible work-flow refinements.
  \end{itemize}
\end{frame}

\begin{frame}{Acknowledgements}
This work was supported in part by Morris Academic Partnership (MAP) and UMN Undergraduate Research Opportunity (UROP).  \\ 

\vspace*{0.2in}

We thank Joe Lane for introducing us to Morse tools and for numerous helpful discussions.
\end{frame}

\begin{frame}
  \frametitle{Discussion}
Questions?
\end{frame}

\end{document}
