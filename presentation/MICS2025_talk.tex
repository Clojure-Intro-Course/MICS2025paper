\documentclass{beamer}
%\usepackage{beamerthemeshadow}
\usetheme{Madrid}
\usepackage{color}
\usepackage[all]{xy}

%% Elena's favorite green (thanks, Fernando!)
\definecolor{ForestGreen}{RGB}{34,139,34}
\definecolor{BlueViolet}{RGB}{138,43,226}
\definecolor{Coquelicot}{RGB}{255, 56, 0}
\definecolor{Teal}{RGB}{2,132,130}
%Uncomment this if you want to show work-in-progress comments
\newcommand{\comment}[1]{{\bf \tt  {#1}}}
% Uncomment this if you don't want to show comments
%\newcommand{\comment}[1]{}
% for color names see https://www.overleaf.com/learn/latex/Using_colors_in_LaTeX
\newcommand{\emcomment}[1]{\textcolor{ForestGreen}{\comment{Elena: {#1}}}}
\newcommand{\todo}[1]{\textcolor{blue}{\comment{To Do: {#1}}}}
\newcommand{\tkcomment}[1]{\textcolor{Teal}{\comment{Tristan: {#1}}}}
\newcommand{\jscomment}[1]{\textcolor{olive}{\comment{Jaydon: {#1}}}}
\newcommand{\jwcomment}[1]{\textcolor{violet}{\comment{John: {#1}}}}
%%%%%%%%%%%%%%%%%%%%%%%%%%%%%%%%%%%%%%%%%%


%%%% choose your presentation style:
\mode<presentation>
{
  \usetheme{Copenhagen} %%%
 \usecolortheme{beaver}

%%% set style for ovelays: lists (and other text) appearing one item at a time
%%% This will create a dimmed preview of next item:
%\setbeamercovered{transparent}
%%% This will hide it entirely:
\setbeamercovered{invisible}
}
%% if you don't want page numbers to show: 
\setbeamertemplate{footline}[page number]{}

%%%% 25 minutes time slot, including questions 

\begin{document}
\title{A beginner-friendly environment for exploring error messages in the Clojure programming language.}
\author{Tristan Kalvoda, Elena Machkasova, Jaydon Stanislowski, and John Walbran}
\institute[UMN Morris] % (optional, but mostly needed)
{
 % \inst{1}%
  University of Minnesota, Morris
}
\date[]  
{Midwest Instruction and Computing Symposium, April 2025}

\begin{frame}
  \titlepage
\end{frame}

\begin{frame}

  \frametitle{Outline}
\tableofcontents
\end{frame}

\section{Overview of Clojure and its Error Messages}

\subsection{Clojure Language and Syntax}
\begin{frame}
\frametitle{Clojure Language and Syntax}
What is Clojure? %- Clojure language and Syntax
\begin{itemize}
  \item Clojure is a part of the Lisp language family
  \begin{itemize}
    \item prefix notation (operators before operands)
    \item expressions are surrounded by parentheses
  \end{itemize}
  Example: \texttt{(/ 9 3)} denotes 9 divided by 3
  \item Clojure is an interpreted language
  \item It's implemented in Java and runs on the Java Virtual Machine (JVM) interpreter
\end{itemize}
  Clojure code \(\rightarrow\) Java code \(\rightarrow\) JVM bytecode \(\rightarrow\)  executed on JVM
\end{frame}

% \begin{frame}
  % \frametitle{Clojure Language and Syntax}
  % \begin{itemize}
    % \item Clojure is an interpreted language
    % \item It's implemented in Java and runs on the Java Virtual Machine (JVM)
    % \begin{itemize}
      % \item executed code compiles to JVM bytecode \emcomment{I corrected the line below slightly. Not sure if you need this line.}
    % \end{itemize}
%     \item Clojure code \(\rightarrow\) Java code \(\rightarrow\) JVM bytecode \(\rightarrow\) \\ executed on JVM
%   \end{itemize}
% \end{frame}

\begin{frame}
  \frametitle{Clojure Language and Syntax}
  Clojure's REPL
  \begin{itemize}
    \item Interactive environment for code evaluation
    \item Read \(\rightarrow\) Evaluate \(\rightarrow\) Print \(\rightarrow\) Loop (REPL)
  \end{itemize}
  \includegraphics{../resources/cljScreenshot.JPG}  
\end{frame}

\subsection{Clojure Error Messages}
\begin{frame}
  \frametitle{Clojure's Error Messages}
  Clojure Exceptions
  \begin{itemize}
    \item All exceptions in Clojure are Java exceptions
    \item Clojure syntax errors will also result in an exception %\emcomment{mention that it is a Java exception}
  \end{itemize}
  Error Messages
  \begin{itemize}
    \item Generate when a exception occurs
    \item Provide error type, cause, and location
  \end{itemize}
\end{frame}

\begin{frame}
    \frametitle{Clojure's Error Messages}
    Anatomy of a Clojure Error Message \\
    \texttt{=> (/ 9 0)} \\
    \texttt{Execution error (ArithmeticException) at user/eval1 (REPL:1).} \\
    \texttt{Divide by zero}
    \begin{itemize}
      \item \texttt{ArithmeticException}: The type of error that occurred
      \item \texttt{user/eval1 (REPL:1)}: The location where the error happened (in this case, REPL, line 1)
      \item \texttt{Divide by zero}: The description of the error's cause
    \end{itemize}
\end{frame}

\section{Babel Project}
\subsection{Motivation and Goals}

\begin{frame}
    \frametitle{Motivation}
    \begin{itemize}
	\item UMN Morris has been teaching functional languages first (Scheme, Racket) to students for over 30 years
         \item Historically done at other major institutions, including MIT
         \item This practice has many pedagogical benefits 
         \item Clojure is good to learn due to its popularity and widespread industry use, but has flaws in its presentation of error messages
    \end{itemize}
\end{frame}

\begin{frame}
    \frametitle{Goals of Babel}
    \begin{itemize}
        \item Designed to be an interactive, beginner-friendly tool for understanding error data in Clojure
        \item Simplifies error messages to be more intuitive, removing jargon and clutter
        \item Utilizes tools to make Clojure errors more descriptive and detailed for formatting messages
    \end{itemize}
\end{frame}

\begin{frame}
    \frametitle{Current State of Babel}
    \begin{itemize}
        \item Our group began working after error replacement was largely achieved
        \item Present goals include exploring more interactive, IDE-like tools that integrate with Babel
        \item Need the capacity to support testing and usability
        \item Lots of refactoring work is being done to support this, as well as refining Babel itself
    \end{itemize}
\end{frame}

\begin{frame}
    \frametitle{Motivation and Goals}
    \textbf{Example} \\
    Consider the error produced by the form below. What does it mean? \\

\vspace*{0.1in}

    \texttt{=> (count 1)} \\
    \texttt{Execution error (UnsupportedOperationException) at user/eval1529 (REPL:1).} \\
    \texttt{count not supported on this type: Long} \\
\end{frame}

\begin{frame}
    \frametitle{On Clojure \textit{spec}}
    Babel is based on an existing Clojure library called \textit{spec}:
    \begin{itemize}
%        \item Tooling to define requirements on properties of a sequence or collection
%        \item Function arguments come in a sequence
        \item Allows specifying requirements on arguments of functions
        \item Requirements are predicates: can check argument types, count, values, etc.
        \item We bind specs to core Clojure functions
        \item If requirements aren't met then a \textit{spec} error happens
        \item Error reports from spec are more detailed than native Clojure error messages
    \end{itemize}
\end{frame}

\subsection{Overview and Usage}
\begin{frame}
    \frametitle{Overview}
   Babel tool:
    \begin{itemize}
        \item Replaces native Clojure error messages
        \item Messages produced by Babel broadly fall into two types:
    \end{itemize}
	Spec errors:
    \begin{itemize}
        \item Babel uses \textit{spec} to specify conditions on Clojure functions
        \item \textit{spec} provides information to make error messages more precise
    \end{itemize}
   Non-spec errors:
    \begin{itemize}
        \item When \textit{spec} cannot be provided, e.g. syntax errors, regex is used to identify error messages
        \item We use a dictionary to rephrase these errors
    \end{itemize}
\end{frame}


\begin{frame}
    \frametitle{Usage}
    \begin{itemize}
        \item Launching a REPL server in the Babel repository allows the tool to ``hook'' to it
	\item Babel can report errors on a loaded Clojure file
	\item Babel catches Clojure errors, including our \textit{spec} reports
	\item Error messages are replaced with modified ones
        %\item All messages displayed in the terminal are generated by Babel rather than Clojure
        \item Currently only supports REPL, not IDE
%/terminal integration, looking to IDE support in the future
        %\item Can also be run on Clojure source code files via the \texttt{(load-file)} function
    \end{itemize}
\end{frame}


\begin{frame}
    \frametitle{Setup}
    \begin{figure}
        \centering
        \includegraphics[width=\textwidth]{../resources/CountError.png}
        \caption{A default error on the \texttt{(count)} function, caused by a mistake that a student typically makes.}
        \label{fig:countError}
    \end{figure}
\end{frame}

\begin{frame}
    \frametitle{Setup}
    \begin{figure}
        \centering
        \includegraphics[width=\textwidth]{../resources/CountBabelError.png}
        \caption{The error message on the same form produced by Babel, containing a description of the potential problem in plain English.}
        \label{fig:BabelCountError}
    \end{figure}
\end{frame}

\section{Interactive Visualizations with Morse}

\subsection{Morse Viewers}

\begin{frame}{Morse}
  \begin{itemize}
    \item<1-> We would like to expand error message handling from REPL strings to interactive visual environments
%Since we have a system for error messages in the REPL, we want to extend to interactive, visual environments
    \item<2-> Joe Lane (Nubank) suggested a new tool called \textit{Morse} 
    \item<3-> Morse is a third party data visualization and navigation tool
    \item<4-> Morse provides a customizable set of viewers for different data structures, like text, maps, or HTML content
    \item<5-> We are developing a custom set of viewers for error messages
    \item<6-> We need to label different parts of error message for different viewers
  \end{itemize}
\end{frame}

%\subsection{Screenshots-Morse}
\begin{frame}{Morse Default}
  \begin{figure}
    \centering
    \includegraphics[height=0.7\textheight]{../resources/MorseDefault.jpg}
    \caption{The default view of the Morse tool.}
    \label{fig:defaultMorse}
  \end{figure}
\end{frame}

\begin{frame}{Morse Lists}
  \begin{figure}
    \centering
    \includegraphics[width=0.8\textwidth]{../resources/MorseSuccessfulForm.png}
    \caption{Morse viewer for vector data. Shows the result of the form \texttt{(map inc [1 2 3])}.}
    \label{fig:MorseVec}
  \end{figure}
\end{frame}


\begin{frame}{Integrating with Morse}
  \begin{itemize}
    \item<1-> Babel is specialized to integrate within a REPL, so it doesn't expose the right hooks for integrating with external tools.
    \item<2-> In order to connect to a Morse session, we introduce a new session layer that exposes communication hooks for other tools.
    \item<3-> \texttt{:init} Behavior on startup, launches a new Morse session connected to the current REPL.
    \item<4-> \texttt{:eval} Behavior on form evaluation. Stores the command and sends it to REPL and Morse.
    \item<5-> \texttt{:caught} Behavior on caught exception. Processes the error in Babel, and changes the Exception$\leftarrow$:wantString processing to instead create a vector of labelled pairs.  
  \end{itemize}
\end{frame}

\begin{frame}{Our Morse Viewer}
  \begin{itemize}
  \item<1-> We used the existing WebView structure within Morse to render error messages.
  \item<2-> We use HTML to add color coding for important terms, and mono-space fonts to highlight code chunks within the message.
  \item<3-> The exact design of the final viewers are still a work in progress.
  \end{itemize}
\end{frame}

\subsection{Prototype Viewer}
\begin{frame}
  \begin{figure}
    \centering
    \includegraphics[width=\textwidth]{../resources/CljDefaultEven.png}
    \caption{The output for the form \texttt{(even?~1~2)} in default Clojure. }
    \label{fig:defaultclj}
  \end{figure}
\end{frame}

\begin{frame}
  \begin{figure}
    \centering
    \includegraphics[width=\textwidth, height=0.25\textheight]{../resources/BabelREPL.jpg}
    \caption{The output for the form \texttt{(even?~1~2)} in Babel through the REPL. }
    \label{fig:BabelREPL}
  \end{figure}
\end{frame}

\begin{frame}
  \begin{figure}
    \centering
    \includegraphics[width=\textwidth]{../resources/BabelViewerExample.png}
    \caption{The output for the form \texttt{(even? 1 2)} in Babel with Morse.}
    \label{fig:defaultclj}
  \end{figure}
\end{frame}

\section{Current State of the Project and Future Work}
\subsection{Current Work}
\begin{frame}{Current Work}
  Work in Progress
  \begin{itemize}
    \item This year's work has involved exploring interactions between Babel and Morse
    \item Babel outputs only strings to REPL, Morse needs labeled data to apply viewers
    \item We are focusing on information flow between REPL and Morse
    \item We are also prototyping models for Morse viewers
  \end{itemize}
\end{frame}

\subsection{Future Work}
\begin{frame}{Future Work}
  We want to:
  \begin{itemize}
    \item finalize data labeling and corresponding viewers
    \item explore interactive capabilities of Morse, e.g. hover text to clarify terms, hyperlinks to documentation
    \item develop Morse viewers for other information, such as the stack trace, and full Java error messages
    \item conduct usability studies with our developments
    \item explore IDE (VS Code) integration for workflow refinements
    %Overview of the VSCode extension development world.
    %https://code.visualstudio.com/api
    %Webview display, allows us to reuse existing viewer code.
    %https://code.visualstudio.com/api/extension-guides/webview
    %Language protocol guide: (LSP, syntax highlighting, etc.)
    %https://code.visualstudio.com/api/language-extensions/overview
  \end{itemize}
  \end{frame}

\begin{frame}{Acknowledgements}
This work was supported in part by Morris Academic Partnership (MAP) and UMN Undergraduate Research Opportunity (UROP).  \\ 

\vspace*{0.2in}

We thank Joe Lane for introducing us to Morse tools and for numerous helpful discussions.
\end{frame}

\begin{frame}
  \frametitle{Discussion}
Questions?
\end{frame}

\end{document}
