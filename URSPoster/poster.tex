%%%%%%%%%%%%%%%%%%%%%%%%%%%%%%%%%%%%%%%%%%%%%%%%%%%%%%%%%%%%%%%
%
% Welcome to Overleaf --- just edit your LaTeX on the left,
% and we'll compile it for you on the right. If you open the
% 'Share' menu, you can invite other users to edit at the same
% time. See www.overleaf.com/learn for more info. Enjoy!
%
%%%%%%%%%%%%%%%%%%%%%%%%%%%%%%%%%%%%%%%%%%%%%%%%%%%%%%%%%%%%%%%
\documentclass[17pt, a0paper, landscape,  margin=0mm, innermargin=10mm, blockverticalspace=3mm, colspace=3mm, subcolspace=3mm]{tikzposter}
\usepackage{graphicx}
\usepackage{authblk}
\usepackage{color}
\usepackage[all]{xy}

\usetheme{board}

%% Elena's favorite green (thanks, Fernando!)
\definecolor{ForestGreen}{RGB}{34,139,34}
\definecolor{BlueViolet}{RGB}{138,43,226}
\definecolor{Coquelicot}{RGB}{255, 56, 0}
\definecolor{Teal}{RGB}{2,132,130}
%Uncomment this if you want to show work-in-progress comments
\newcommand{\comment}[1]{{\bf \tt  {#1}}}
% Uncomment this if you don't want to show comments
%\newcommand{\comment}[1]{}
% for color names see https://www.overleaf.com/learn/latex/Using_colors_in_LaTeX
\newcommand{\emcomment}[1]{\textcolor{ForestGreen}{\comment{Elena: {#1}}}}
\newcommand{\todo}[1]{\textcolor{blue}{\comment{To Do: {#1}}}}
\newcommand{\tkcomment}[1]{\textcolor{Teal}{\comment{Tristan: {#1}}}}
\newcommand{\jscomment}[1]{\textcolor{olive}{\comment{Jaydon: {#1}}}}
\newcommand{\jwcomment}[1]{\textcolor{violet}{\comment{John: {#1}}}}

\title{A beginner-friendly environment for exploring error messages in the Clojure programming language.}
\author[1]{John Walbran}
\author[1]{Jaydon Stanislowski}
\author[1]{Tristan Kaldova}
\author[1]{Elena Machkasova}
\date{\today}
\affil[1]{Division of Science and Mathematics, University of Minnesota, Morris}
\usetheme{Envelope}
\usecolorstyle{Australia}

\begin{document}

\makeatletter
\def\maketitle{\AB@maketitle}
\makeatother

\maketitle

\block{Overview}{
Programmers are imperfect, and will often make mistakes when programming and create a program error, for example, attempting to divide by zero. When a computer tries to run a program with an error, the program will halt and present the details of the error to the user in the form of an error message. These error messages are often very jargon-heavy, and are not designed to be palatable to a novice programmer. This creates significant friction for new programmers trying to learn programming languages. This work is a part of an ongoing project (called Babel) led by Elena Machkasova in an attempt to ease this friction in the Clojure programming language. Currently, Babel software is able to replace standard error messages with ones that are more helpful for a beginner audience. My contribution to this project is an exploration of potential tools to effectively display information about errors in an interactive and intuitive manner. The most promising of these tools up to this point has been Morse, created by the company Cognitect, owned by Nubank. As this project continues to explore the possibilities of Morse and how it can integrate with the existing Babel system, we are putting together potential setups that novice programmers can use to effectively understand and explore the causes of the errors they come across. This project presents the setups that have been developed and discuss their benefits and tradeoffs in helping novice programmers understand error messages.
\emcomment{Needs to be shortened and changed to match this year's work}

}
\begin{columns}
\column{0.3}
        \block{Overview of Functional Programming and Clojure}{
%\tkcomment{Tristan's section}
    \begin{itemize}
%	\item \emcomment{Includes reasons to teach FP to beginners}
	\item Functional programming has a rich history of being introduced early in programming education.
	\item It emphasizes breaking problems into smaller, manageable, easy-to-combine pieces.
%promotes a more declarative programming style. Which is where you tell it what you want to happen rather than giving it a list of steps to follow.
	\item It builds a strong foundation for students in subjects like recursion, higher-order functions, immutable data, and problem decomposition.
    \item Clojure is a functional 
%Lisp-based language 
that encourages these principles and is a great tool for introducing students to functional programming.
    \item It has a simple, minimal syntax with prefix notation in statements and statements are surrounded by parenthesis.
    \item It uses an interactive Read-Eval-Print-Loop (REPL) environment: the user types some code, the system evaluates it and prints the answer.
    \end{itemize}
}

 \block{Clojure Error Messages}{
%\tkcomment{Tristan's section}
    \begin{itemize}
    \item Errors in Clojure are Java exceptions, as Clojure code is compiled into Java code. This means that syntax errors in Clojure will also result in an exception.
\emcomment{The target audience doesn't know the difference.}
    \item Error messages generate when an exception occurs and provide error type, cause, and location.
    \item Error messages are designed for more experienced developers and may have certain terminology that does not make sense to beginner programmers.
    \item Example of an error message produced when the \texttt{even?} function, which expects a single input, is given two instead:
    \end{itemize}
    \texttt{user=> (even? 2 3)} \\
    \texttt{Execution error (ArityException) at user/eval1 (REPL:1).} \\
    \texttt{Wrong number of args (2) passed to: clojure.core/even?}
}

\block{Overview of Babel.}{
\jscomment{Jaydon's section.}
    \begin{itemize}
        \item We wanted to build upon existing tools that would likely be maintained in the future.
        \item Maintaining a tool like Babel from completely ourselves is unfeasible for a small research group, so we instead wanted to modify existing tools. 
        \item Previous work on this project created a tool, Babel that would provide error message replacement as a text response to running the program. 
        \item While the Babel messages were an improvement, beginners need different information at different points in their learning, so a static message can't cater to beginners at every point in their learning.
    \end{itemize}
}

\block{Goals for This Year.}{
\jscomment{Jaydon's section.}
    \begin{itemize}
        \item Most of the work this year was searching for established tools that would accomplish what we need.
        \item We had previously explored tools that were not being maintained, creating dead ends.
        \item After consulting with developers currently in the industry, we were pointed towards Morse. 
        \item Morse is a data visualization tool designed for professional development and debugging, which provides a strong skeleton for extending Babel [2].
        \item We successfully integrated Morse with Babel, not only allowing us to display the error message from Babel, but also have the details about the error displayed in an interactive way.
        \item We were able to also include embedded links to external information regarding the error, such as the official documentation.
    \end{itemize}
    }

\column{0.4}
    \block{Examples}{
%\jwcomment{John's section.}
%\emcomment{(even? "two"), (even? 2 3)}

    Default Clojure: \\
    \texttt{user=>(even? "two")} \\
    \texttt{Execution error (IllegalArgumentException) at user/eval2044 (REPL:1).}\\
    \texttt{Argument must be an integer: two}


    \texttt{user=>(even? 2 3)} \\
    \texttt{Execution error (ArityException) at user/eval2046 (REPL:1).} \\ 
     \texttt{Wrong number of args (2) passed to: clojure.core/even?} \\

    Babel (in REPL): \\
    \texttt{user=>(even? "two")}\\
    \texttt{The first argument of (even? "two") was expected to be a number but is a string "two" instead.}\\
    \texttt{In Clojure interactive session on line 1.}


    \texttt{user=>(even? 2 3)} \\
    \texttt{Wrong number of arguments in (even? 2 3): the function even? expects one argument but was given two arguments.}\\
    \texttt{In Clojure interactive session on line 1.} \\

    Babel (with Morse): \\
    \includegraphics[width=\linewidth]{figures/ErrorViewer1Cropped.png}
    \includegraphics[width=\linewidth]{figures/ErrorViewer2Cropped.png}
}


\column{0.3}
\block{Progress This Year.}{
%\jwcomment{John's section.}
    \begin{itemize}
        \item We improved internal tooling to allow us to easily select different types of error messages and test how they are presented.
        \item We implemented a skeleton framework for data flow to the interactive viewer.
        \item We created a system for labeling parts of an error message to be able to use different formats for different parts.
        \item We created a simple viewer to format error messages based on labeled data.
    \end{itemize}
}

\block{Future Work.}{
\jscomment{Jaydon's section.}
}

\block{Acknowledgments}{
 Thanks to Morris Academic Partnership (MAP) and the Undergratuate Research Opprotunity (UROP) for funding this work. \\
Thanks to Joe Lane (Nubank) for guidance on tools for this project. 
}

\block{Sources}{
    \begin{itemize}
        \item {[1]} \emph{clojure.org}
        \item {[2]} Morse, Nubank \emph{https://github.com/nubank/morse}
	\item {[3]} Tao Dong, Kandarp Khandwala,  \textit{The Impact of ``Cosmetic'' Changes on the Usability of Error Messages}, 2019 CHI Conference on Human Factors in Computing Systems.
	\item {[4]} Matthias Felleisen, Robert Bruce Findler, Matthew Flatt, and Shriram Krishnamurthi \textit{The Structure and Interpretation of the Computer Science Curriculum},  Journal of Functional Programming, 2004.
    \end{itemize}
}
\block{}{
\begin{tikzfigure}
    \includegraphics[width=\linewidth]{figures/UMM logo.png}
\end{tikzfigure}
}
\end{columns}


\end{document}