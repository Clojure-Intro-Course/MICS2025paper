% This is sigproc-sp.tex -FILE FOR V2.6SP OF ACM_PROC_ARTICLE-SP.CLS
% OCTOBER 2002
%
% It is an example file showing how to use the 'acm_proc_article-sp.cls' V2.6SP
% LaTeX2e document class file for Conference Proceedings submissions.
% 
%----------------------------------------------------------------------------------------------------------------
% This .tex file (and associated .cls V2.6SP) *DOES NOT* produce:
%       1) The Permission Statement
%       2) The Conference (location) Info information
%       3) The Copyright Line with ACM data
%       4) Page numbering
%
%  However, both the CopyrightYear (default to 2002) and the ACM Copyright Data
% (default to X-XXXXX-XX-X/XX/XX) can still be over-ridden by whatever the author
% inserts into the source .tex file.
% e.g.
% \CopyrightYear{2003} will cause 2003 to appear in the copyright line.
% \crdata{0-12345-67-8/90/12} will cause 0-12345-67-8/90/12 to appear in the copyright line.
%
%
%---------------------------------------------------------------------------------------------------------------
% It is an example which *does* use the .bib file (from which the .bbl file
% is produced).
% REMEMBER HOWEVER: After having produced the .bbl file,
% and prior to final submission,
% you need to 'insert'  your .bbl file into your source .tex file so as to provide
% ONE 'self-contained' source file.
%
% Questions regarding SIGS should be sent to
% Adrienne Griscti ---> griscti@acm.org
%
% Questions/suggestions regarding the guidelines, .tex and .cls files, etc. to
% Gerald Murray ---> murray@acm.org 
%
% For tracking purposes - this is V2.6SP - OCTOBER 2002


\documentclass[12pt]{article}

\setlength{\oddsidemargin}{0in}
\setlength{\evensidemargin}{0in}
\setlength{\topmargin}{0in}
\setlength{\headheight}{0in}
\setlength{\headsep}{0in}
\setlength{\textwidth}{6in}
\setlength{\textheight}{9in}
\setlength{\parindent}{0in} 

\usepackage{graphicx} %For jpg figure inclusion
\usepackage{times} %For typeface
\usepackage{epsfig}
\usepackage{xcolor} %For Comments
%\usepackage[all]{xy}
\usepackage{float}
%\usepackage{subfigure} 
\usepackage{hyperref}
\usepackage{url}
\usepackage{parskip}

%% Elena's favorite green (thanks, Fernando!)
\definecolor{ForestGreen}{RGB}{34,139,34}
\definecolor{BlueViolet}{RGB}{138,43,226}
\definecolor{Coquelicot}{RGB}{255, 56, 0}
\definecolor{Teal}{RGB}{2,132,130}
%Uncomment this if you want to show work-in-progress comments
\newcommand{\comment}[1]{{\bf \tt  {#1}}}
% Uncomment this if you don't want to show comments
%\newcommand{\comment}[1]{}
% for color names see https://www.overleaf.com/learn/latex/Using_colors_in_LaTeX
\newcommand{\emcomment}[1]{\textcolor{ForestGreen}{\comment{Elena: {#1}}}}
\newcommand{\todo}[1]{\textcolor{blue}{\comment{To Do: {#1}}}}
\newcommand{\tkcomment}[1]{\textcolor{Teal}{\comment{Tristan: {#1}}}}
\newcommand{\jscomment}[1]{\textcolor{olive}{\comment{Jaydon: {#1}}}}
\newcommand{\jwcomment}[1]{\textcolor{violet}{\comment{John: {#1}}}}
%%%%%%%%%%%%%%%%%%%%%%%%%%%%%%%%%%%%%%%%%%

\begin{document}
\pagestyle{plain}
%
% --- Author Metadata here ---
%\conferenceinfo{WOODSTOCK}{'97 El Paso, Texas USA}
%\setpagenumber{50}
%\CopyrightYear{2002} % Allows default copyright year (2002) to be
%over-ridden - IF NEED BE. 
%\crdata{0-12345-67-8/90/01}  % Allows default copyright data
%(X-XXXXX-XX-X/XX/XX) to be over-ridden. 
% --- End of Author Metadata ---

\title{A beginner-friendly environment for exploring error messages in the Clojure programming language.}
%\subtitle{[Extended Abstract \comment{DO WE NEED THIS?}]
%\titlenote{}}
%
% You need the command \numberofauthors to handle the "boxing"
% and alignment of the authors under the title, and to add
% a section for authors number 4 through n.
%
% Up to the first three authors are aligned under the title;
% use the \alignauthor commands below to handle those names
% and affiliations. Add names, affiliations, addresses for
% additional authors as the argument to \additionalauthors;
% these will be set for you without further effort on your
% part as the last section in the body of your article BEFORE
% References or any Appendices.

\author{
Tristan Kalvoda, Elena Machkasova, Jaydon Stanislowski, and John Walbran\\
Computer Science Discipline \\
University of Minnesota Morris\\
Morris, MN 56267\\
kalvo007@umn.edu, elenam@umn.edu, stani152@umn.edu, walbr042@umn.edu
}
\date{}
\maketitle
\thispagestyle{empty}
%\alcomment{Should these say @morris.umn.edu?}

\section*{\centering Abstract}


\newpage
\setcounter{page}{1}

\section{Introduction}

\emcomment{Sample Comment} \\
\tkcomment{Sample Comment} \\
\jscomment{Sample Comment} \\
\jwcomment{Sample Comment}

The Clojure programming language, built on the Java Virtual Machine, is a widely used functional language. 
Its simple syntax makes it a powerful language to introduce beginner programmers to key concepts of programming.
However, debugging in Clojure can be unintuitive because Clojure errors are exceptions in the underlying Java language. 
They frequently contain terminology intended for experienced developers, leaving out important context. 
As a result, what would otherwise be a powerful educational tool presents a difficult learning curve for new programmers.

In this paper we discuss the structure of Clojure error messages, 
explain the data flow for the error messages processing, 
and the chosen data representation. 
We outline options for formatting the error messages to beginners, as well as the principles that guide our choices. 
We present design examples of views and interactions for a variety of error messages. 
We conclude with a discussion of key open questions and future directions. 

\subsection{Related Work}\label{subsec:related}

\emcomment{Talking about existing literature, such as Racket's errors, and other people?}
\jscomment{Discussion of what existing literature is relevant and why?}

\section{Background}\label{sec:background}

\subsection{Clojure and its Error Messages}\label{subsec:clojure-errs}
\tkcomment{Tristan's subsection}

As an example of one such mistake, a novice to Clojure might write an expression like
\begin{verbatim}
(filter neg? 1 -1 0)
\end{verbatim}
instead of a correct application to a vector:
\begin{verbatim}
(filter neg? [1 -1 0])
\end{verbatim}
 Clojure presents this error as an {\it ArityException}, referring to the number of arguments of the function. 
 The message is cryptic for beginners, especially without the documentation.

\subsection{Clojure Spec}\label{subsec:spec}
\emcomment{Elena's section.}

Clojure \textit{spec} is an addition to Clojure first released around 2016.  \emcomment{add a citation for https://clojure.org/about/spec}
It is a set of features that allow setting specification, in particular for functions and macros, that are checked before the function call or macro expansion, respectively. 
These specification typically include the number and types of arguments, although the may include other conditions.
The flexible nature of spec allows programmers to choose what parameters to check and, to a degree, how would the failure manifest itself. 

As an example, one may specify that a function \texttt{f} takes no fewer than 2 and no more than 4 arguments, the first argument must be a non-negative integer, and the second one must be sequence. 
Types (or any other conditions) are specified as predicates. 
For example, we can require that the first argument of \texttt{f} satisfies both \texttt{number?} and \texttt{pos?} predicates, and the second satisfies \texttt{seq?}
Any existing Clojure predicate can be used, or the user can write their own. 

If arguments not satisfying the spec are passed to the function, a spec error will occur. 
A spec error is an exception of a specific type \texttt{clojure.lang.ExceptionInfo} that contains data describing the error. 
The data include the failed predicate.
For example, if the second argument of the function 
\texttt{f} above is not a sequence, it fails the predicate \texttt{seq?}
 


\section{Babel Project}\label{sec:babel}
\subsection{Overview of the Project}\label{subsec:overview}
\jscomment{Jaydon's subsection}
Our research project, Babel, aims to provide beginner programmers with a tool that replaces 
Clojure error messages with clearer explanations, removing unfamiliar jargon.

Our previous versions of Babel transformed error messages into a beginner-friendly 
format by taking an exception object and producing a plaintext description of the issue. 
It employs a Clojure feature called spec that provides a way to specify the number and types of arguments allowed for a function. 
Unlike default error messages, spec allows us to display the erroneous code and the specific argument that violates the requirements. 
Using spec, the error message above is presented as: 
\begin{verbatim}
(filter neg? 1 -1 0)
\end{verbatim}
``Wrong number of arguments in (filter neg? 1 -1 0): the function filter expects one or two arguments but was given four arguments.''

Our project currently includes specs for many core functions and also rewrites common non-spec errors. 
However, one ongoing problem is to display the improved error messages in an IDE or an interactive panel. 

Our project poses the following core research question: 
``How can we design error message presentation and interactions to help beginners fix errors in their programs?'' 
Exploring this question requires having a robust environment for experimenting with interactive error messages.

\subsection{Tentative subsection: Babel Control Flow and Use of Spec}\label{subsec:control-flow}
\emcomment{Do we need a subsection on control flow in babel itself and how spec is used?}

\subsection{Classifying Error Messages}\label{subsec:classification}
\tkcomment{Tristan's subsection}


\section{Adding Visual Interactive Environment}\label{sec:interactive}
\jwcomment{This is where we discuss Morse, why we chose it, and what it offers}
\jwcomment{This also talks about what exceptions actually have as data, and then discuss how we 
represent things.}

Our goal is to improve the user experience by using Morse, a third-party viewer, to present error information in a more beginner-friendly format. 
For instance, visual cues will help distinguish between code and values, which are indistinguishable by default. 
Babel will also adjust the level of detail based on context, for example, displaying only relevant parts of a stack trace. 
Babel will also provide links to Clojure documentation where applicable.

\subsection{Connecting to the REPL}\label{subsec:connect-to-REPL}

Usage of babel starts with a standard Clojure REPL run in the user's project, with babel loaded as a library.
Then, in order to start a babel session, the babel library needs to be imported into the REPL, and the function 
\texttt{(run-in-repl)} needs to be run, launching a Morse viewer, and running a babel REPL.
\jwcomment{The above command is likely going to be renamed, and this assumes a packaged build of babel, which
doesn't quite exist yet. (But we've shown we can, so I'm going to assume this exists at point of submission.)}
The \texttt{(run-in-repl)} function starts by creating a child REPL on top of the current REPL, adding hooks that connect babel to 
the existing REPL environment. 
These hooks add behavior and allow for callback functions on standard actions, such as form evaluation and raised exceptions.
We then use the data caught by these hooks to pass into Morse to be displayed within the current viewer template. 
When an exception occurs, the following gets passed to Morse:
\begin{itemize}
	\item{The form that caused the error.}
	\item{The line number and file of the error.}
	\item{The vector of labelled pairs forming the error message.}
	\item{The url of relevant core documentation (when applicable).}
\end{itemize}

\subsection{Designing Morse Viewers}\label{subsec:Morse-Viewers}
Morse allows for the creation of custom views, which act as visual layouts for incoming data. 
We use this extendability to create custom viewers, which allow us to create and tailor the visualization of the error message data.


\jwcomment{John's subsection}

\subsection{Designing an Error Messages Interface}\label{subsec:interface}
\emcomment{TBD}


\emcomment{Need at least one citation for LaTeX to work:}
\cite{Hickey:2008}

		
\section{Current Abilities and Future Work}\label{sec:conclusion}
	\subsection{Current Abilities}\label{subsec:current-abilities}
	\emcomment{TBD}
	

	\subsection{Future work}\label{subsec:future}
	\emcomment{Talk about the usability study + other things - Elena's section}

\bibliographystyle{acm}
\bibliography{mics2025}

\end{document}
