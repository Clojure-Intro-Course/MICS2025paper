% This is sigproc-sp.tex -FILE FOR V2.6SP OF ACM_PROC_ARTICLE-SP.CLS
% OCTOBER 2002
%
% It is an example file showing how to use the 'acm_proc_article-sp.cls' V2.6SP
% LaTeX2e document class file for Conference Proceedings submissions.
% 
%----------------------------------------------------------------------------------------------------------------
% This .tex file (and associated .cls V2.6SP) *DOES NOT* produce:
%       1) The Permission Statement
%       2) The Conference (location) Info information
%       3) The Copyright Line with ACM data
%       4) Page numbering
%
%  However, both the CopyrightYear (default to 2002) and the ACM Copyright Data
% (default to X-XXXXX-XX-X/XX/XX) can still be over-ridden by whatever the author
% inserts into the source .tex file.
% e.g.
% \CopyrightYear{2003} will cause 2003 to appear in the copyright line.
% \crdata{0-12345-67-8/90/12} will cause 0-12345-67-8/90/12 to appear in the copyright line.
%
%
%---------------------------------------------------------------------------------------------------------------
% It is an example which *does* use the .bib file (from which the .bbl file
% is produced).
% REMEMBER HOWEVER: After having produced the .bbl file,
% and prior to final submission,
% you need to 'insert'  your .bbl file into your source .tex file so as to provide
% ONE 'self-contained' source file.
%
% Questions regarding SIGS should be sent to
% Adrienne Griscti ---> griscti@acm.org
%
% Questions/suggestions regarding the guidelines, .tex and .cls files, etc. to
% Gerald Murray ---> murray@acm.org 
%
% For tracking purposes - this is V2.6SP - OCTOBER 2002


\documentclass[12pt]{article}

\setlength{\oddsidemargin}{0in}
\setlength{\evensidemargin}{0in}
\setlength{\topmargin}{0in}
\setlength{\headheight}{0in}
\setlength{\headsep}{0in}
\setlength{\textwidth}{6in}
\setlength{\textheight}{9in}
\setlength{\parindent}{0in} 

\usepackage{graphicx} %For jpg figure inclusion
\usepackage{times} %For typeface
\usepackage{epsfig}
\usepackage{xcolor} %For Comments
%\usepackage[all]{xy}
\usepackage{float}
%\usepackage{subfigure} 
\usepackage{hyperref}
\usepackage{url}
\usepackage{parskip}
\usepackage{listings}

%% Elena's favorite green (thanks, Fernando!)
\definecolor{ForestGreen}{RGB}{34,139,34}
\definecolor{BlueViolet}{RGB}{138,43,226}
\definecolor{Coquelicot}{RGB}{255, 56, 0}
\definecolor{Teal}{RGB}{2,132,130}
%Uncomment this if you want to show work-in-progress comments
\newcommand{\comment}[1]{{\bf \tt  {#1}}}
% Uncomment this if you don't want to show comments
%\newcommand{\comment}[1]{}
% for color names see https://www.overleaf.com/learn/latex/Using_colors_in_LaTeX
\newcommand{\emcomment}[1]{\textcolor{ForestGreen}{\comment{Elena: {#1}}}}
\newcommand{\todo}[1]{\textcolor{blue}{\comment{To Do: {#1}}}}
\newcommand{\tkcomment}[1]{\textcolor{Teal}{\comment{Tristan: {#1}}}}
\newcommand{\jscomment}[1]{\textcolor{olive}{\comment{Jaydon: {#1}}}}
\newcommand{\jwcomment}[1]{\textcolor{violet}{\comment{John: {#1}}}}
%%%%%%%%%%%%%%%%%%%%%%%%%%%%%%%%%%%%%%%%%%

\begin{document}
\pagestyle{plain}
%
% --- Author Metadata here ---
%\conferenceinfo{WOODSTOCK}{'97 El Paso, Texas USA}
%\setpagenumber{50}
%\CopyrightYear{2002} % Allows default copyright year (2002) to be
%over-ridden - IF NEED BE. 
%\crdata{0-12345-67-8/90/01}  % Allows default copyright data
%(X-XXXXX-XX-X/XX/XX) to be over-ridden. 
% --- End of Author Metadata ---

\title{A beginner-friendly environment for exploring error messages in the Clojure programming language.}
%\subtitle{[Extended Abstract \comment{DO WE NEED THIS?}]
%\titlenote{}}
%
% You need the command \numberofauthors to handle the "boxing"
% and alignment of the authors under the title, and to add
% a section for authors number 4 through n.
%
% Up to the first three authors are aligned under the title;
% use the \alignauthor commands below to handle those names
% and affiliations. Add names, affiliations, addresses for
% additional authors as the argument to \additionalauthors;
% these will be set for you without further effort on your
% part as the last section in the body of your article BEFORE
% References or any Appendices.

\author{
Tristan Kalvoda, Elena Machkasova, Jaydon Stanislowski, and John Walbran\\
Computer Science Discipline \\
University of Minnesota Morris\\
Morris, MN 56267\\
kalvo007@umn.edu, elenam@umn.edu, stani152@umn.edu, walbr042@umn.edu
}
\date{}
\maketitle
\thispagestyle{empty}
%\alcomment{Should these say @morris.umn.edu?}

\section*{\centering Abstract}


\newpage
\setcounter{page}{1}

\section{Introduction}

%\emcomment{Sample Comment} \\
%\tkcomment{Sample Comment} \\
%\jscomment{Sample Comment} \\
%\jwcomment{Sample Comment}

The Clojure programming language is a functional language in the Lisp family, developed and released in 2007 
by Rich Hickey~\cite{Hickey:2008}. 
Clojure is implemented in Java, compiles to Java bytecode, and runs in the Java Virtual Machine (JVM), 
allowing programmers to use Java libraries directly from Clojure. 
However, unlike Java, it is a dynamically typed functional language with immutable data as a default (mutable data 
must be specifically declared).
Clojure quickly became known in industry. Even though the number of Clojure programmers is not a large percentage
of all developers (1.3\% of professional developers reported extensive use of Clojure), it remains one of the most admired languages (68.5\%) and is the 3rd highest-paying programming language, according to the 2024 Stack Overflow Developers Survey~\cite{survey}.

There is a long history of teaching Lisp to beginner programmers, starting from the famous  ``Structure and Interpretation of Computer Programs'' curriculum~\cite{Abelson} and continuing with the introduction of the 
Racket programming language that provides a tiered system of Lisp-based beginner languages~\cite{Felleisen:2004,racket}. While Clojure is also a Lisp language and shares the benefits of focusing on development of functions that follow the structure of the data representation, some of Clojure's aspects make it more challenging than other Lisps for novice programmers. 

One such aspect is Clojure error messages: due to Clojure implementation via the underlying Java language, 
Clojure errors are Java exceptions. 
They frequently contain terminology intended for experienced developers, leaving out important context. 
As a result, what would otherwise be a powerful educational tool presents a difficult learning curve for new programmers.

\emcomment{A 2-sentence overview of Babel}

\emcomment{Need to fill out the section references}
In this paper we discuss the structure of Clojure error messages, 
explain the data flow for the error messages processing, 
and the chosen data representation. 
We outline options for formatting the error messages to beginners, as well as the principles that guide our choices. 
We present design examples of views and interactions for a variety of error messages. 
We conclude with a discussion of key open questions and future directions. 



\subsection{Related Work}\label{subsec:related}

\cite{cosmetic}

\emcomment{Talking about existing literature, such as Racket's errors, and other people?}
\jscomment{Discussion of what existing literature is relevant and why?}

\section{Background}\label{sec:background}

\subsection{Clojure and its Error Messages}\label{subsec:clojure-errs}
%\tkcomment{Tristan's subsection}
%\emcomment{I am removing the list: the paper should use a narratve. \\}
%\begin{itemize}
    %\item \textbf{Exception handling takes from the JVM} 
	Clojure is a hosted language, running on the Java Virtual Machine (JVM), which is a platform \emcomment{an interpeter} that allows a program to run on any computer without needing to be rewritten for each one. 
	The JVM works by compiling code into bytecode, which it can then run. When Clojure code is executed, it compiles into JVM bytecode, allowing it to use many of the features that the JVM provides, 
	but most importantly for the work presented here is that it uses the Java exception system for handling errors and when an exception is thrown in Clojure, it is a Java exception.
    
   % \item \textbf{What are Exceptions} \\
   An exception is an event or error that disrupts the normal flow of a program’s execution. In Clojure, exceptions typically occur when something goes wrong, such as dividing by 
   zero or passing an inappropriate argument to a function. It’s important to note here that syntax errors will also result in an exception, because during the compilation of Clojure code into JVM bytecode, 
   any syntax error will prevent the code from being compiled and cause an exception.

   Error messages are generated when an exception occurs. They provide information about
   the type of error as well as where it happens. Let’s look at an example:

	%\item \textbf{Error Messages} \\
	\begin{figure}[h]
		\centering
		\begin{lstlisting}[breaklines=true, basicstyle=\ttfamily]
	user=> (/ 9 0)
	Execution error (ArithmeticException) at user/eval1 (REPL:1).
	Divide by zero
		\end{lstlisting}
	\end{figure}
\emcomment{Also briefly explain the actual cause of the error and how the Clojure wording relates to it.}	

What’s going wrong here?
	\begin{itemize}
		\item \texttt{ ArithmeticException } is the type of error that happens.
		\item \texttt{ user/eval1 (REPL:1) } which is where the Clojure code is being typed and tested and it happened in the first line.
		\item \texttt{ Divide by zero } is the given description for the cause of the error
	\end{itemize}

	\tkcomment{finish the Error Object and Ex-triage parts}
	\tkcomment{Error messages here need to be changed to the ones for (/9 0)}
	%\item \textbf{Error Object} \\
\emcomment{You are explaining this to those who have never used Clojure (and maybe not even Java). This is very cryptic. I also think that some names of fields aren't important.}
	The error message is a part of an exception object that contains more information,
	most notably a stack longer stack trace and nested errors in \texttt{:via} if there are any. You can get the most recent error message by using \texttt{*e}
	\begin{figure}[h]
		\centering
		\begin{lstlisting}[breaklines=true, basicstyle=\ttfamily]
#error {
	:cause "Divide by zero"
	:via
	[{:type java.lang.ArithmeticException
	  :message "Divide by zero"
	  :at [clojure.lang.Numbers divide "Numbers.java" 190]}]
	:trace
	[[clojure.lang.Numbers divide "Numbers.java" 190]
	... omitting 18 lines...
	 [clojure.main main "main.java" 40]]}

		\end{lstlisting}
	\end{figure}
	

	%\item \textbf{Ex-triage} \\
	We use the exception object in the \texttt{ex-triage} function to help interpret exceptions, which returns an analysis of the phase, error, cause, and location of an error.
	 the phase in the \texttt{ex-triage} lets us know when the exception occurs.
	
	For example, in the earlier case of the divide by zero error, the exception occurs in the execution phase

	\tkcomment{: and line from :line are being seperated by line wrapping}

	\begin{figure}[h]
		\centering
		\begin{lstlisting}[breaklines=true, basicstyle=\ttfamily]
user=> (clojure.main/ex-triage(Throwable->map *e))
#:clojure.error{:class java.lang.ArithmeticException, :line 1, :cause "Divide by zero", :symbol user/eval1, :phase :execution}
		\end{lstlisting}
	\end{figure}

	\begin{figure}[h]
		\centering
		\begin{tabular}{|c|l|}
			\hline
			\textbf{Phase} & \textbf{Description} \\
			\hline
			:read-source & Error thrown while reading characters at the REPL or from a source file. \\
			:compile-syntax-check & Syntax error caught during compilation. \\
			:execution & Any errors thrown at execution time. \\
			:read-eval-result & Error thrown while reading the result of execution. \\
			:print-eval-result & Error thrown while printing the result of execution. \\
			\hline
		\end{tabular}
%\end{itemize}
	\end{figure}
	

As an example of one such mistake, a novice to Clojure might write an expression like
\begin{verbatim}
(filter neg? 1 -1 0)
\end{verbatim}
instead of a correct application to a vector:
\begin{verbatim}
(filter neg? [1 -1 0])
\end{verbatim}
 Clojure presents this error as an {\it ArityException}, referring to the number of arguments of the function. 
 The message is cryptic for beginners, especially without the documentation.

\subsection{Clojure Spec}\label{subsec:spec}
%\emcomment{Elena's section.}
Clojure \textit{spec} is an addition to Clojure first released around 2016~\cite{spec-overview}.
It is a set of features that allow setting specification, in particular for functions, that are checked before the function call. 
These specification typically include the number and types of arguments, although they may include other conditions.
The flexible nature of spec allows programmers to choose what parameters to check and, to a degree, how would the failure manifest itself. 

As an example, one may specify that a function \texttt{f} takes no fewer than 2 and no more than 4 arguments, the first argument must be a non-negative integer, and the second one must be sequence. 
Types (or any other conditions) are specified as predicates. 
For example, we can require that the first argument of \texttt{f} satisfies both \texttt{number?} and \texttt{pos?} predicates, and the second satisfies \texttt{seq?} (a predicate that determines if a Clojure object is a sequence).
Any existing Clojure predicate can be used, or the user can write their own. 

If arguments not satisfying the spec are passed to the function, a spec error will occur. 
A spec error is an exception of a specific type \texttt{clojure.lang.ExceptionInfo} that contains data describing the error. 
The data include the failed predicate.
For example, if the second argument of the function 
\texttt{f} above is a number $5$, and not a sequence, the error message states that it failed the predicate \texttt{seq?}
Even more importantly, the argument that's causing the error (in this case $5$) will be included in the spec error message, whereas without using spec the error message would be a Java \texttt{ClassCastException} that doesn't report the value of the failing argument, only the types that it failed to convert to each other. 
The more detailed information allows us to leverage spec to make error messages more understandable to novice programmers, as described in Section~\ref{subsec:overview}. 

\section{Babel Project}\label{sec:babel}
\subsection{Overview of the Project}\label{subsec:overview}
\jscomment{Jaydon's subsection}

Our research project, Babel, is a tool intended to aid novice programmers in confusions that may arise from the highly specialized design of Clojure's error messages. It does this by providing learners with substitute messages that describe code problems in simpler terms, removing unfamiliar jargon and low-level clutter. For example, the error message produced via the function call below:

\jscomment{TODO: Please fix the figure number reference}

\begin{figure}[h]
	\centering
	\begin{lstlisting}[breaklines=true, basicstyle=\ttfamily]
        user=> (count 1)
        Execution error (UnsupportedOperationException) at user/eval1529 (REPL:1).
        count not supported on this type: Long
	\end{lstlisting}
\end{figure}

is transformed into a new message with an improved description of the error's cause, replacing the reference to the internal type \verb|Long| and removing details about the Java exception object and the temporary interpreter file where the error occurred:

\begin{figure}[h]
	\centering
	\begin{lstlisting}[breaklines=true, basicstyle=\ttfamily]
        babel.middleware=> (count 1)
        Function count does not allow a number as an argument in this position.

        In Clojure interactive session on line 1.
        Call sequence:
        [Clojure interactive session (repl)]
	\end{lstlisting}
\end{figure}

Babel constructs its error messages by intercepting exception objects directly from the Clojure interpreter, called the REPL (Read - Evaluate - Print - Loop), hooking an observer to a running instance of a REPL server.

The data from the exception object is triaged in order to identify the type of error that occurred, checking components of the error such as its type, message, and, if it's a spec error, the spec data associated with it.
\jscomment{I'd like to be more specific about the technical details of how (modify-message) categorizes the errors, but I'm not sure if that's necessary for this section or if it would be feasible}

We are currently exploring ways to improve the categorization of error messages and optimize the paths an exception object must take to be properly analyzed and rewritten. \jscomment{(Tristan's work?)}

\jscomment{Spec was already introduced in the last section}
Babel employs a Clojure feature called spec that provides a way to specify the number and types of arguments allowed for a function. 
Unlike default error messages, spec allows us to display the erroneous code and the specific argument that violates the requirements. 
Using spec, the error message above is presented as: 
\begin{verbatim}
(filter neg? 1 -1 0)
\end{verbatim}
``Wrong number of arguments in (filter neg? 1 -1 0): the function filter expects one or two arguments but was given four arguments.''

Our project currently includes specs for many core functions and also rewrites common non-spec errors. 

However, one ongoing problem is to display the improved error messages in an IDE or an interactive panel. 

Our project poses the following core research question: 
``How can we design error message presentation and interactions to help beginners fix errors in their programs?'' 
Exploring this question requires having a robust environment for experimenting with interactive error messages.

\subsection{Tentative subsection: Babel Control Flow and Use of Spec}\label{subsec:control-flow}
\emcomment{Do we need a subsection on control flow in babel itself and how spec is used?}

\subsection{Classifying Error Messages}\label{subsec:classification}
\begin{itemize}
    \item \textbf{Spec Errors vs. Non-Spec Errors} \\
    We separate out the spec errors and the non-spec errors.

    \item \textbf{Error Messages Classification} \\
    Error messages can be classified by things like:
    \begin{itemize}
        \item Error Types
        \item Level of Nesting
        \item Phase
    \end{itemize}

    \item \textbf{Babel Messages vs. Messages from Dictionary} \\
    \begin{itemize}
        \item LazySeq messages not caught in spec because worried that catching and printing at the point will change the order of eval (these are caught in the print phase) -- key decisions we made with our spec.
    \end{itemize}

    \item \textbf{Documenting Errors} \\
    Error on the receiving end and how to get where it came from.

    \item \textbf{How to Get Useful Information from Error?} \\
    Reverse-engineered through things like nested errors.

\end{itemize}
\tkcomment{Tristan's subsection}

\section{Adding Visual Interactive Environment}\label{sec:interactive}
\emcomment{Copied from the abstract:}
Our goal with Babel is to improve the user experience by using the a third-party Morse viewer to present error information in a better, more beginner-friendly format. 
For instance, visual cues will help distinguish between strings and numbers, which can be indistinguishable in standard Clojure error Messages. 
Babel will also adjust the level of detail based on the error’s context. 
For example, a simple number being passed as a string may not print out a full stack trace, instead just focusing on type mismatch. 
Babel will also provide relevant links to Clojure documentation on clojuredocs.org where applicable.

\jwcomment{This is where we discuss Morse, why we chose it, and what it offers}
\jwcomment{This also talks about what exceptions actually have as data, and then discuss how we 
represent things.}

Our goal is to improve the user experience by using Morse, a third-party viewer, to present error information in a more beginner-friendly format. 
For instance, visual cues will help distinguish between code and values, which are indistinguishable by default. 
Babel will also adjust the level of detail based on context, for example, displaying only relevant parts of a stack trace. 
Babel will also provide links to Clojure documentation where applicable.

\subsection{Connecting to the REPL}\label{subsec:connect-to-REPL}

Usage of babel starts with a standard Clojure REPL run in the user's project, with babel loaded as a library.
Then, in order to start a babel session, the babel library needs to be imported into the REPL, and the function 
\texttt{(run-in-repl)} needs to be run, launching a Morse viewer, and running a babel REPL.
\jwcomment{The above command is likely going to be renamed, and this assumes a packaged build of babel, which
doesn't quite exist yet. (But we've shown we can, so I'm going to assume this exists at point of submission.)}
The \texttt{(run-in-repl)} function starts by creating a child REPL on top of the current REPL, adding hooks that connect babel to 
the existing REPL environment. 
These hooks add behavior and allow for callback functions on standard actions, such as form evaluation and raised exceptions.
We then use the data caught by these hooks to pass into Morse to be displayed within the current viewer template. 
When an exception occurs, the following gets passed to Morse:
\begin{itemize}
	\item{The form that caused the error.}
	\item{The line number and file of the error.}
	\item{The vector of labelled pairs forming the error message.}
	\item{The url of relevant core documentation (when applicable).}
\end{itemize}

\jwcomment{Make more explicit comment about vector labelling.}

\jwcomment{There will be a diagram for this later.}

\subsection{Designing Morse Viewers}\label{subsec:Morse-Viewers}
Morse allows for the creation of custom views, which act as visual layouts for incoming data. 
We use this extendability to create custom viewers, which allow us to create and tailor the visualization of the error message data.


\jwcomment{John's subsection}

\subsection{Designing an Error Messages Interface}\label{subsec:interface}
\emcomment{TBD}

		
\section{Current Abilities and Future Work}\label{sec:conclusion}
	\subsection{Current Abilities}\label{subsec:current-abilities}
	\emcomment{TBD}
	

	\subsection{Future work}\label{subsec:future}
	\emcomment{Talk about the usability study + other things - Elena's section}

\bibliographystyle{acm}
\bibliography{mics2025}

\end{document}
